% Preliminary results

%%%%%%%%%%%%%%%%
%%% Preamble %%%
%%%%%%%%%%%%%%%%
\documentclass[10pt]{article}

% Encoding & Font
\usepackage[utf8]{inputenc}                    % encoding
\usepackage[T1]{fontenc}                       % encoding
\usepackage{lmodern}                           %
%% or
%\usepackage{cmbright}
%\renewcommand{\familydefault}{\sfdefault}      % font

% Bibliography
\usepackage[round]{natbib}                     % omit 'round' option if you prefer square brackets
\bibliographystyle{plainnat}                   % style of bibliography

% Math
\usepackage{amsmath}                           % for all elaborate math 
\usepackage{amssymb}                           % for \mathbb{R} = real numbers
\usepackage{dsfont}                            % for \mathds{1} = identity matrix
%\usepackage{mathtools}                         % further math stuff

% Figures
\usepackage{graphicx}
\usepackage[font=small]{caption}                           % more options for caption  
\usepackage[font=footnotesize]{subcaption}                        % subfloats
\graphicspath{{../figures/}}                    % set path for figures
\DeclareGraphicsExtensions{.pdf, .png, .jpg, .eps}     % extension of figures; priority = order of list

% Set language
\usepackage[american]{babel}				   % american / ngerman	
\usepackage{hyperref}                          % hyperref e.g. to other documents; autoref

%% Table
%\newcommand{\myfloatalign}{\centering} % to be used with each float for alignment
%\usepackage{diagbox, pict2e}    % \diagbox
%\usepackage{colortbl}    % \rowcolor  
%\usepackage{multirow}           % \multirow{n}{lcr}{content}
%\definecolor{TableColor}{gray}{0.8}
%%% Right aligned table cells with fixed length; e.g. x{2.5cm}
%\newcolumntype{x}[1]{%
%>{\raggedleft\hspace{0pt}}p{#1}}%
%\newcommand{\tn}{\tabularnewline}
%\newcommand{\tnn}{\tabularnewline[0.1cm]} % newline for costume table cells 
%\let\mc\multicolumn
%% Table title sections
%\newcommand{\paramsheadline}[1]{\rowcolor{TableColor}\multicolumn{9}{l}{\spacedlowsmallcaps{#1}}}
%\newcommand{\modelheadline}[1]{\rowcolor{TableColor}\multicolumn{2}{l}{\spacedlowsmallcaps{#1}}}

\usepackage{tabu}


%%%%%%%%%%%%
%%% Main %%%
%%%%%%%%%%%%
\begin{document}

\section{Introduction}
\label{sec:introduction}

We are interested in the accuracy of the dimensional reduction proposed by Gao et al.
In order to determine the accuracy, we define an error as the distance between a reduced 
fixed point and the fixed point of the reduced system. We then calculate the expected value and the 
fluctuation of this error analytically for a system of generalized Lotka-Volterra equations 
and specific random interaction matrices $A$. 

Define errors in $x$ and $\beta$ direction:

\begin{align}
    \mathrm{err}_x &= 1 - \frac{x(\beta_\mathrm{eff})}{x_\mathrm{eff}} \,, \\
    \mathrm{err}_\beta &= 1 - \frac{\beta(x_\mathrm{eff})}{\beta_\mathrm{eff}} \,.
\end{align}

For the GLV they coincide since  the fixed point in the reduced system is

\begin{equation}
    x(\beta_\mathrm{eff}) = - \frac{\alpha}{\beta_\mathrm{eff}} \,,
\end{equation}

such that both errors become

\begin{equation}
    \mathrm{err} = 1 + \frac{\alpha}{x_\mathrm{eff} \beta_\mathrm{eff}} \,.
\end{equation}

Inserting the definitions of $x_\mathrm{eff}$ and $\beta_\mathrm{eff}$ from Gao et al.'s paper, 

\begin{align}
    x_\mathrm{eff} &= \frac{\sum_{ij} A_{ij} x_j}{\sum_{mn} A_{mn}} \,, \\  
    \beta_\mathrm{eff} &= \frac{\sum_{ijk} A_{ij} A_{jk} }{\sum_{mn} A_{mn}},
\end{align}

we get 

\begin{equation}
    \begin{split}
        \mathrm{err} 
            &= 1 + \frac{\alpha \left(\sum_{mn} A_{mn}\right) \left(\sum_{mn} A_{mn}\right)   
                }{- \alpha \left( \sum_{ijk} A_{ij} {A^{-1}}_{jk} \right) \left(\sum_{ijk} A_{ij} A_{jk}\right)}  \\ 
            &= 1 - \frac{\sum_{ijkl} A_{ij} A_{kl} }{ S \sum_{ijk} A_{ij} A_{jk} } \\
            &= 1 - \frac{n}{d} \,,
    \end{split}
\end{equation}

where we defined

\begin{align}
    n &:= \sum_{ijkl} A_{ij}A_{kl} \,, \\
    d &:= S \cdot \sum_{ijk} A_{ij}A_{jk} \,, 
\end{align}

and the $A_{ij}$ are the entries of the interaction matrix $A$.

In order to calculate the expected value and variance analytically, 
we make one key assumption, namely that the expected values of numerator and denominator 
of the error and its square can be taken independently of each other. This approximation 
should work well for strictly positive entries $A_{ij} > 0$ with vanishing probability density 
close to 0. For $A_{ij}$ not following this constraint, the actual analytical expected value is
expected to explode. Note that this behaviour may not be observed when sampling $A_{ij}$, because 
the probability associated with the denominator being zero may be very small. 

Making this approximation, we get

\begin{equation}
     \mathbb{E}[\mathrm{err}] = \mathbb{E}\left[1 - \frac{n}{d}\right] \approx 1 - \frac{\mathbb{E}[n]}{\mathbb{E}[d]} \,,
\end{equation}

and for the variance

\begin{equation}
    \label{eq:Eerr}
 \mathrm{Var}(\mathrm{err}) = \mathbb{E}\left[\left(\mathrm{err} - \mathbb{E}[\mathrm{err}]\right)^2\right]
 = \mathbb{E}\left[\left(
 \frac{d \mathbb{E}[n] - n \mathbb{E}[d]}{d \mathbb{E}[d]}
 \right)^2\right]
\end{equation}

we have
 
\begin{equation}
    \label{eq:varerr}
    \mathrm{Var}(\mathrm{err}) \approx 
    \frac{\mathbb{E}[d^2]\mathbb{E}[n]^2 - 2 \mathbb{E}[nd]\mathbb{E}[n] \mathbb{E}[d] + \mathbb{E}[n^2]\mathbb{E}[d]^2}
    {\mathbb{E}[d^2]\mathbb{E}[d]^2} \,.
\end{equation}

By automatizing the combinatorial problems associated with the sums, we calculate both values, as described in the next section.

\section{Description of the algorithm}
\label{sec:description_of_the_algorithm}

As stated above, we want assess the expected value \autoref{eq:Eerr} 
and variance \autoref{eq:varerr} of the error. Hence, we need to calculate the terms

\begin{align}
    \mathbb{E}[d]         &= S \cdot \mathbb{E}\left[ \sum_{ija} A_{ia}A_{aj} \right] \,; \\ 
    \mathbb{E}[n]         &= \mathbb{E} \left[ \sum_{ijkl} A_{ij}A_{kl} \right]  \,;   \\ 
    \mathbb{E}[d^2]       &= S^2 \cdot \mathbb{E} \left[ \sum_{ijklab} A_{ia}A_{aj}A_{kb}A_{bj} \right]  \,;   \\
    \mathbb{E}[n \cdot d] &= S \cdot \mathbb{E} \left[ \sum_{ijklmna} A_{ij}A_{kl}A_{ma}A_{an} \right]  \,;  \\
    \mathbb{E}[n^2]       &= \mathbb{E} \left[ \sum_{ijklmnop} A_{ij}A_{kl}A_{mn}A_{op} \right]   
\end{align}

where all indices are iterated over $\{1, 2, ..., S\}$. In the simplest version of interaction matrix $A$, 
the entries $A_{ij}$ are all i.d.d. and we need to separate out pairs $A_{ij}^2$ as they will lead to contributions 
other than $\mu^2$, where $\mu = \mathbb{E}[A_{ij}]$ (and similarly for higher order tuples).
In order to do so, we devised an algorithm which is described for the simple example of $\mathbb{E}[n]$, 
which may even be calculated by hand quite easily. It consists of the following steps: 

\begin{enumerate}
    \item \textbf{Generation of a feasible index combination.}
        Such a combination like $(i = j \ne k \ne l, i \ne l)$ is encoded in the upper triangle of a $4 \times 4$ 
        boolean matrix, which in this case would look like  

        \begin{equation}
        P = \begin{bmatrix}
          - & T & F & F\\
          - & - & F & F\\
          - & - & - & F\\
          - & - & - & - 
         \end{bmatrix}  \,.
        \end{equation}
         
        Here, $P_{ij} = T$ implies that the $i$-th index is equal to the $j$-th index ($F$ for $\ne$), numbering
        the indices as $(i, j, k, l) \sim (0, 1, 2, 3)$. We dump non-feasible combinations such as   

        \begin{equation}
        P_\mathrm{illegal} = \begin{bmatrix}
          - & T & F & F\\
          - & - & F & T\\
          - & - & - & F\\
          - & - & - & - 
         \end{bmatrix} \,.
        \end{equation}
     
    \item\textbf{Calculating the number of terms $n_P$ associated with $P$.}
        In order to do so, we sort the indices into cliques containing only equal ones. The number of cliques, called multiplicity factor $m_P$ is directly related to the number of terms by

        \begin{equation}
         n_P = S (S - 1) ... (S - m_P) = \prod_{\alpha = 0}^{m_P - 1} (S - \alpha)   \,.
        \end{equation}

    \item\textbf{Project the matrix $P$ down to a $2 \times 2$ boolean matrix $R_P$.} 
        This matrix encodes whether the actual random variables $X = A_{ij}$ and $Y = A_{kl}$ are equal. 
        We have $X = Y$ only if $i = k$ and $j = l$, which in terms of the matrices $P$ and $R_P$ is
        stated as

        \begin{equation}
            R_{01}(P) = P_{02} \land P_{13} \,.
        \end{equation}
                
    \item\textbf{Calculating actual cliques $\mathcal{C}(R(P))$.}
        In this case, the only possible cliques are $XY$ or $X^2$, trivially encoded by $R_{01}(P)$. 

    \item\textbf{Add contribution to overall sum.}
        Each clique is associated with a specific factor arising from taking the expected value. 
        In this case, we only have  

        \begin{equation}
            \mathbb{E}[XY] = \mu^2  \qquad \text{and} \qquad \mathbb{E}[X^2] = \sigma^2 + \mu^2 \,.  
        \end{equation}

        We sum each term multiplied by the multiplicity $n_P$, iterating over all feasible index matrices $P$
        to get the corresponding expected value

        \begin{equation}
            \mathbb{E}[d] = S \cdot \sum_P n_P \, \mathbb{E}[\mathcal{C}(R(P))] \,.
        \end{equation}

\end{enumerate}

To incorporate correlation of transposed entries, we make use of the lower triangle of $R$. 
In this example, we have $A_{ij} = X$ and $A_{kl} = A_{ji} = X^T$ only if $i = l$, $j = k$ 
and $(i, j) \ne (k, l)$, that is

\begin{equation}
    R_{10}(P) = P_{03} \land P_{12} \land \lnot R_{01}(P) \,.
\end{equation}
 
The associated clique has the expected value

\begin{equation}
    \mathbb{E}[X X^T] = \rho \sigma^2 + \mu^2 \,.
\end{equation}

If we finally want to allow for the diagonal elements to be drawn from a different distribution, 
we employ the free diagonal in a similar manner. In this example:

\begin{equation}
    R_{00}(P) = P_{01}, \qquad \qquad R_{11}(P) = P_{23} \,,  
\end{equation}

and the associated expected values

\begin{equation}
    \mathbb{E}[XY_d] = \mu\mu_d, \qquad \mathbb{E}[X_dY_d] = \mu_d^2, \qquad \mathbb{E}[X_d^2] = \sigma_d^2 + \mu_d^2\,.
\end{equation}

For the second order terms, $R$ becomes a $4 \times 4$ boolean matrix, and the full set of possible cliques
becomes both larger (30 terms for both correlation and specific diagonal) and more complicated 
(incorporating higher moments such as $\mathbb{E}[X^3Y]$). 


\section{Results}
\label{sec:results}

The analytical expressions for expected value and variance of the error
for different cases of interaction matrices $A$ are listed in \autoref{tab:results}.
We approximated the expressions to the highest order in the network size $S$, 
as we want to look at large networks.

We summarize these results by making the following observations:
\begin{itemize}
        \item 
        In all cases, the error (or its fluctuations) grows without bound if the ratio $\frac{\mu}{\sigma}$ goes to zero 
 for a given network size $S$.
 
\item The order of the fluctuations (namely $S^{-\frac{3}{2}}$) remains the same for all cases,
    while the order of the expected value changes. Thus,
 \begin{itemize}
     \item  for interaction matrices $A$ without correlation ($\rho = 0$), the term dominating the error for large $S$ 
         are the fluctuations while the mean value is either zero (for iid entries $A_{ij}$)
         or of order $S^{-2}$ (in case of a constant diagonal);
     \item for networks with non-zero correlation, the mean becomes the dominating term of order $S^{-1}$.  
 \end{itemize}

 \item  If the diagonal is of the same scale as $S$, the error may explode. This happens if
     $A_{ii} = - d_c$, where $d_c = (S - 1) \mu$  corresponds to the value of $d$ where the 
     interaction matrix becomes stable and non-reactive for positive $\mu$. 
\end{itemize}   

In order to test these analytical results, we sampled the interaction matrix with
the corresponding statistics numerically and compared the empirical mean 
and standard deviation with the theoretical predictions. 
The results can be observed in \autoref{fig:comparison}. 
In all cases, the theoretical prediction is met very well. 
There is a notable but small deviation for small network sizes  $S = 20$, 
namely slight underestimation of the mean for the case of correlation, 
cf. \autoref{fig:rho_03}and \autoref{fig:rho_-07}. 

Note that for the case of a constant diagonal close to the critical value,
shown in \autoref{fig:vary_d_rho_0}, 
the theoretical value is not expected to give a good 
approximation to the empirical average, since in this case, 
the expected value of the denominator $\mathbb{E}[d]$ becomes 
zero. In this case, the approximation of taking numerator 
and denominator separately is not justified. Furthermore, 
sampling becomes difficult, as outliers may govern the empirical 
mean and standard deviation. 


\begin{table}[h]
    \centering
    \caption[Results]{Resulting analytical expressions, approximated to highest order in $S$. 
        }
    \label{tab:model_description}
    \small
    {\tabulinesep=1.2mm
       \begin{tabu} {b{4cm} p{2.5cm} p{4.5cm}}
            \textbf{Case} & $\mathbb{E}[\mathrm{err}]$ & $\mathrm{Var}(\mathrm{err})$ \\
           \hline
    $A_{ij}$ i.i.d.                    & $0$ (exact)                     & $\frac{\sigma^4}{S^3 \mu^4}$ \\
    \vspace{1.mm}

    Correlation: &&\\
    $\rho = \mathrm{corr}(A_{ij}, A_{ji}) \in [-1, 1]$     
        & $\frac{\rho \sigma^2}{S\mu^2}$ 
        & $\frac{\sigma^4}{S^3 \mu^4} \left(2\frac{\mu^2}{\sigma^2} \rho + (\rho - 1)^2\right) $ \\ 
    \vspace{1.mm}

    Constant diagonal: &&\\
    $A_{ii} = -d$ of order $1$ or $\sqrt{S}$    
        & $\frac{\sigma^2\left((S - 2)\rho - 1\right)}{S^2\mu^2}$ 
        & $\frac{\sigma^4}{S^3 \mu^4} \left(2\frac{\mu^2}{\sigma^2} \rho + (\rho - 1)^2\right) $ \\ 
    $\qquad$ for $\rho = 0$   
        & -$\frac{\sigma^2}{S^2\mu^2}$ 
        & $\frac{\sigma^4}{S^3 \mu^4} $ \\
    $\qquad$ for $\rho \ne 0$ 
        & $\frac{\sigma^2\rho}{S\mu^2}$ 
        & $\frac{\sigma^4}{S^3 \mu^4} \left(2\frac{\mu^2}{\sigma^2} \rho + (\rho - 1)^2\right) $ \\ 

    $A_{ii} = -d$ of order $S$ 
        & $\frac{\sigma^2\left((S - 2)\rho - 1\right)}{S^2\mu^2\left(\frac{d}{d_c} - 1\right)^2}$ 
        & $\frac{\sigma^4}{S^3 \mu^4\left(\frac{d}{d_c} - 1\right)^4}\left(2\frac{\mu^2}{\sigma^2} \rho + (\rho - 1)^2\right)$ \\
    $\qquad$ for $\rho = 0$   
        & -$\frac{\sigma^2}{S^2\mu^2\left(\frac{d}{d_c} - 1\right)^2}$ 
        & $\frac{\sigma^4 }{S^3 \mu^4\left(\frac{d}{d_c} - 1\right)^4} $  \\
    $\qquad$ for $\rho \ne 0$ 
        & $\frac{\sigma^2\rho}{S\mu^2\left(\frac{d}{d_c} - 1\right)^2}$ 
        & $\frac{\sigma^4}{S^3 \mu^4\left(\frac{d}{d_c} - 1\right)^4}\left(2\frac{\mu^2}{\sigma^2} \rho + (\rho - 1)^2\right)$ \\ \hline
    \label{tab:results}
   \end{tabu}}
\end{table}

\begin{figure}[tb]
    \begin{subfigure}{.47\textwidth}
      \centering
        \includegraphics[width=1.0\linewidth]{error_GLV_A_iid}
        \caption{ 
            All elements i.i.d.
        }
        \label{fig:A_iid}
    \end{subfigure}%
    \begin{subfigure}{.47\textwidth}
        \centering
        \includegraphics[width=1.0\linewidth]{error_GLV_rho_03}
        \caption{
            Positive correlation:
            $\rho = 0.3$. 
            %$A_{ij} \sim \mathcal{N}(\mu = 1, \sigma^2 = 1)$, 
        }
        \label{fig:rho_03}
    \end{subfigure} 
\vspace{0.5cm}
\\
    \begin{subfigure}{.47\textwidth}
      \centering
        \includegraphics[width=1.0\linewidth]{error_GLV_rho_-07}
        \caption{
            Negative correlation:
            $\rho = -0.7$. 
            %$A_{ij} \sim \mathcal{N}(\mu = 1, \sigma^2 = 1)$, 
        }
        \label{fig:rho_-07}
    \end{subfigure}
    \begin{subfigure}{.47\textwidth}
      \centering
        \includegraphics[width=1.0\linewidth]{error_GLV_d_2_rho_0}
        \caption{
            Constant diagonal $A_{ii} = -2 d_c, \rho = 0$.
        }
        \label{fig:d_2_rho_0}
    \end{subfigure}
\vspace{0.5cm}
\\
    \begin{subfigure}{.47\textwidth}
        \centering
        \includegraphics[width=1.0\linewidth]{error_GLV_vary_rho_no_d}
        \caption{
            Varying correlation, no specific diagonal.
        }
        \label{fig:vary_rho_no_d}
    \end{subfigure}
    \begin{subfigure}{.47\textwidth}
        \centering
        \includegraphics[width=1.0\linewidth]{error_GLV_vary_d_rho_0}
        \caption{
            Varying constant diagonal, no correlation.
        }
        \label{fig:vary_d_rho_0}
    \end{subfigure}

    \caption{Comparison of theoretical results with numerical samples. 
        Each blue dot corresponds to the error calculated for one 
        interaction matrix. We sampled 500 matrices independently and 
        calculated empirical means $\mu_\mathrm{err}$ and standard deviations $\sigma_\mathrm{err}$
        (plotted as red and green dots, respectively).
        The theoretical mean $\mathbb{E}[\mathrm{err}]$ is plotted as a red line, 
        the shaded area indicates the predicted standard deviation $\mathrm{std}(\mathrm{err})$. 
    For all figures, $\mu = \sigma = 1$, network size for (e) and (f): $S = 50$.
}
    \label{fig:comparison}
\end{figure}

% More figures
\begin{figure}[tb]
    \begin{subfigure}{.47\textwidth}
      \centering
        \includegraphics[width=1.0\linewidth]{error_GLV_vary_mu_sigma_rho_0}
        \caption{ 
        Varying mean $\mu$ for $\sigma = 0.5, \rho = 0$.
        }
        \label{fig:vary_mu_sigma_rho_0}
    \end{subfigure}%
    \begin{subfigure}{.47\textwidth}
      \centering
        \includegraphics[width=1.0\linewidth]{error_GLV_vary_mu_sigma_rho_02}
        \caption{ 
        Varying mean $\mu$ for $\sigma = 0.5, \rho = 0.2$.
        }
        \label{fig:vary_mu_sigma_rho_02}
    \end{subfigure}%
%\vspace{0.5cm}
%\\

    \caption{Comparison of theoretical results with numerical samples, part 2.
}
    \label{fig:comparison}
\end{figure}


%%%%%%%%%%%%%%%%%%%%%%%%%%%%%%%%%%%%%%%%%%%%%%%%%%%%%%5


%%%%%%%%%%%%%%%%%%%%
%%% Bibliography %%% 
%%%%%%%%%%%%%%%%%%%%
%\newpage
\pagestyle{empty}
\bibliography{../database}

\end{document}
